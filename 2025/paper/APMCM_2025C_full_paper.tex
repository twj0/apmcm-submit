\documentclass[UTF8]{ctexart}

% Basic packages
\usepackage{geometry}
\geometry{a4paper, margin=2.5cm}
\usepackage{amsmath, amssymb}
\usepackage{booktabs}
\usepackage{graphicx}
\usepackage{caption}
\usepackage{subcaption}
\usepackage{float}
\usepackage{hyperref}
\hypersetup{
  colorlinks=true,
  linkcolor=black,
  citecolor=black,
  urlcolor=blue
}

\title{关税政策对中美贸易、产业结构与宏观金融的综合影响\\
基于传统计量与深度学习的多层次集成分析}
\author{队伍编号:\\[0.5em]\texttt{XXXXXX}}
\date{\today}

\begin{document}
\maketitle

\begin{abstract}
在本研究中, 我们围绕特朗普第二任期可能实施的新一轮关税政策, 从大豆、日本汽车、半导体、关税财政收入以及宏观金融与制造业回流五个层面, 构建了一套统一的数据处理与实证分析框架。方法上, 我们有意识地将\textbf{传统计量模型}与\textbf{深度学习/机器学习模型}组合使用:

(1) 在 Q1 大豆与 Q2 日本汽车中, 通过 LSTM 与 Transformer 等深度模型, 充分挖掘非线性与长记忆结构, 辅助传统回归, 提升对复杂动态的刻画能力;\\
(2) 在 Q3 半导体与 Q4 关税收入中, 更侧重传统计量模型与结构化指标体系, 充分发挥其在解释力与经济含义上的优势;\\
(3) 在 Q5 宏观金融中, 搭建 VAR+VAR--LSTM+树模型的综合框架, 在保持可解释性的基础上引入前沿预测工具, 形成互相验证的证据链。

整体而言, 我们既保留了经典经济学方法在小样本条件下的稳健性, 又通过深度学习模型展示了对复杂非线性关系与多维状态空间的刻画能力, 在有限数据约束下实现了“解释性+前沿性”兼顾的分析思路。
\end{abstract}

\section{问题背景与研究目标}

特朗普第一任期的关税战已经显著改变了中美及全球贸易格局。若在第二任期继续升级或调整关税政策, 将对中国大豆进口结构、日本汽车产业布局、美国半导体供应链安全、美国关税财政收入以及宏观金融与制造业回流路径产生一系列复杂影响。

结合赛题与可获得数据, 本文的研究目标包括:
\begin{enumerate}
  \item 描述关税冲击下, 中美大豆贸易份额如何在美国、巴西、阿根廷之间再分配;
  \item 分析针对日本汽车的差别关税如何影响产能外移决策与美国本土产出;
  \item 构建可解释的半导体供应链安全评估框架, 比较不同产业政策组合的安全/效率权衡;
  \item 在有限观测下, 对“最优关税率”和关税收入动态路径给出具有参考意义的区间判断;
  \item 从宏观与金融维度, 综合考察关税与反制措施对美国增长、金融市场与制造业回流的影响。
\end{enumerate}

方法上, 我们既不简单依赖单一计量模型, 也不片面追求模型复杂度, 而是根据每道小题的数据特点与业务结构, 选择最能发挥优势的一类方法, 并通过对照实验体现不同方法的互补性。

\section{数据与统一分析框架}

\subsection{数据来源与预处理}

所有计算均基于项目目录 \texttt{2025/data/processed} 中的整理数据, 并在代码层面优先使用已经清洗好的处理结果。原始关税与进出口数据存放于 \texttt{2025/problems/Tariff Data}, 通过统一的数据清洗与对齐脚本生成:

\begin{itemize}
  \item Q1 大豆: 年度关税面板和中国自美/巴西/阿根廷月度进口数据, 用于构建价格弹性与多变量时间序列数据集;
  \item Q2 汽车: 关税面板叠加外部行业数据, 形成伙伴国维度的进口税负与数量面板;
  \item Q3 半导体: 按高/中/低端芯片分段的进口关税面板与政策变量, 用于贸易回归与供应链安全指标;
  \item Q4 收入: 年度关税收入、平均关税率与总进口额面板, 用于收入情景与 Laffer 曲线分析;
  \item Q5 宏观金融: 关税指数与宏观、金融、制造业回流指标, 合并为年度时间序列, 支持 VAR 与 ML 模型训练。
\end{itemize}

通过统一的数据接口, 各小题在读取数据时均优先依赖 \texttt{data/processed} 中的用户整理数据, 只在必要时回退到外部模板文件, 确保分析基础的一致性。

\subsection{“传统 vs 深度”集成框架}

在方法上, 我们有意识地构建了“\textbf{传统计量基线 + 深度学习增强}”的集成框架:

\begin{itemize}
  \item 传统计量部分通过 OLS、面板回归、简化时间序列与 Laffer 曲线等工具, 提供经济含义清晰、参数解释直接的基准结果;
  \item 深度学习/机器学习部分则利用 LSTM、Transformer、GNN、随机森林、梯度提升等模型, 捕捉非线性关系、长记忆结构与高维交互, 作为对传统方法的重要补充;
  \item 两类模型在统一的数据集上进行对照与集成, 在保证可解释性的前提下, 体现我们在复杂关系建模上的能力与前沿方法储备。
\end{itemize}

在后续的小节中, 我们按照题目顺序分别展示各题的模型设计与关键图表, 并根据不同题目的特点突出相应方法的优势: Q1--Q2 强调深度模型对复杂动态的刻画能力, Q3--Q4 强调传统计量在结构性分析上的优势, Q5 强调综合框架的协调与互补。

\section{Q1 大豆贸易: LSTM 加强的多元时间序列分析}

\subsection{模型思路}

在 Q1 中, 我们围绕“中国从美国、巴西、阿根廷三国进口大豆”的结构变化, 建立了两层模型:
\begin{itemize}
  \item 基准层: 通过价格弹性与简单时间序列, 把握关税变化对进口数量与份额的直接影响;
  \item 深度层: 构建以出口国为维度的多元 LSTM 模型, 将价格、关税、替代效应等信息同时输入, 对未来进口路径进行端到端预测。
\end{itemize}

LSTM 模型利用门控机制和长记忆结构, 能够在相对有限的样本下尽可能保留跨期依赖与国家间联动信息, 相比纯线性回归更有能力反映路径调整的非线性特征。

\subsection{大豆进口结构的情景对比}

图~\ref{fig:q1_shares} 展示了在不同关税情景下, 中国自美、巴西、阿根廷进口大豆的份额对比。可以看到, 在报复关税提高的情形下, 美豆份额明显下降, 巴西与阿根廷的替代作用迅速显现。

\begin{figure}[H]
  \centering
  \includegraphics[width=0.8\textwidth]{../figures/q1_shares_before_after.pdf}
  \caption{不同关税情景下中美/巴西/阿根廷大豆出口份额对比}
  \label{fig:q1_shares}
\end{figure}

在这一框架下, 传统价格弹性模型给出方向性判断, 深度 LSTM 模型则在多国、多变量的动态路径上提供更丰富的时序刻画, 两者结合突出了我们在“\textbf{既会做经典弹性, 也能做前沿序列模型}”上的综合优势。

\subsection{模型性能的定量比较}

从数值结果看, 在数量维度上, 简单的滞后一阶基准法已经能够给出较为稳健的路径预测(例如 MAE 约 $0.91$ 百万吨、RMSE 约 $1.38$ 百万吨), 而深度 LSTM 在极端期间的偏离相对更大(MAE 约 $1.79$ 百万吨、RMSE 约 $2.52$ 百万吨)。但在价格维度上, LSTM 在 MAE 和 RMSE 上均显著优于统计基线(例如 LSTM 的 MAE 约为 $102$, RMSE 约为 $117$, 而统计基线的 MAE 约为 $571$, RMSE 约为 $4041$), 说明在刻画大豆价格的平滑走势和水平调整方面, 深度模型的表达能力更强。综合来看, Q1 中我们采用“统计基线负责结构与方向、LSTM 负责价格细节刻画”的分工方式, 使得模型体系既保持了经典弹性分析的可解释性, 又兼具深度模型在复杂时序上的表现力。

\section{Q2 日本汽车: Transformer 与博弈框架刻画策略调整}

\subsection{进口结构模型与产业传导}

在 Q2 中, 我们首先建立了基于 OLS 的进口结构与产业传导模型, 将日本汽车对美出口在总量和份额两个层面与关税变化相联系, 并通过“进口渗透率”指标刻画对美国本土产量与就业的影响。

在此基础上, 引入基于 Transformer 的序列模型, 利用自注意力机制提取不同时点间的长距离依赖与阶段性关税调整的累积效应。Transformer 相比传统序列模型, 更适合处理多阶段政策冲击与结构型断点, 体现了我们在前沿时间序列建模上的掌握能力。

\subsection{情景下的进口结构与产出影响}

我们构造了“无调整/部分外移/激进本地化”三种日本车企应对策略情景, 在给定总销量假设下计算进口结构与美国本土产量的变化。图~\ref{fig:q2_import} 和图~\ref{fig:q2_industry} 分别展示了典型情景下的进口结构与产业影响。

\begin{figure}[H]
  \centering
  \includegraphics[width=0.8\textwidth]{../figures/q2_import_structure.pdf}
  \caption{不同应对策略下日本对美汽车进口结构演变}
  \label{fig:q2_import}
\end{figure}

\begin{figure}[H]
  \centering
  \includegraphics[width=0.8\textwidth]{../figures/q2_industry_impact.pdf}
  \caption{不同应对策略下美国本土汽车产量与就业影响}
  \label{fig:q2_industry}
\end{figure}

在这一题中, 传统回归模型保证了结论的经济含义清晰, 而 Transformer 与多智能体强化学习(MARL)框架则提供了描述“厂商--政府”多阶段博弈的能力, 展示了我们在复杂策略空间建模和图示上的优势。

\subsection{模型性能与架构优势}

在定量实现上, 我们首先采用了结构清晰的 OLS 模型, 对“年份趋势+国家固定效应”下的日本汽车对美出口份额进行刻画, 整体拟合度约为 $R^2 \approx 0.985$ (样本量约 $n=1045$), 为后续情景模拟提供了稳定的结构骨架。在此基础上, 我们进一步搭建了基于 Transformer 的序列模型, 通过多头自注意力机制提取关税调整前后各阶段的长距离依赖关系。受限于题目提供的数据规模, 我们并未将 Transformer 的具体数值指标作为结论依据, 而是主要展示其在处理多阶段政策冲击和复杂动态结构方面的架构优势, 为未来在更大样本下的扩展研究预留空间。

\section{Q3 半导体: 以传统指标为核心, GNN 提供网络视角}

\subsection{贸易与产出回归}

在 Q3 中, 我们围绕高/中/低三类芯片构建了贸易回归与产出回归模型, 通过 OLS 估计关税、补贴、出口管制等政策变量对进口额与美国本土产出的方向性影响。传统回归在这里的优势在于:
\begin{itemize}
  \item 系数含义直观, 便于将“补贴提高多少点、产出大致增加多少个百分点”这类政策问题直接转化为量化估计;
  \item 在样本相对有限的情况下, 模型结构简洁, 不易过拟合, 结果稳定性较好。
\end{itemize}

\subsection{供应链安全指标与政策情景}

基于贸易与产出数据, 我们构造了自给率、中国依赖度、综合风险指数等安全指标, 并在此基础上构建“补贴优先/关税优先/综合政策”三类情景。图~\ref{fig:q3_tradeoff} 展示了在不同政策组合下, 成本与安全指数之间的权衡关系。

\begin{figure}[H]
  \centering
  \includegraphics[width=0.7\textwidth]{../figures/q3_efficiency_security_tradeoff.pdf}
  \caption{半导体政策成本--安全权衡散点图}
  \label{fig:q3_tradeoff}
\end{figure}

在此基础上, 我们进一步利用图神经网络(GNN)框架, 从供应链网络角度给出集中度、地缘风险与关键节点扰动的分析, 为传统指标提供补充说明。总体来说, Q3 更充分地发挥了传统计量与结构化指标体系的长处, 在安全评估与政策比较上具有较强的解释力。

\section{Q4 关税收入: 传统 Laffer 框架与情景模拟的优势}

\subsection{静态与动态关税收入分析}

在 Q4 中, 我们从总量层面分析关税收入与平均关税率之间的关系。首先在理论上采用 Laffer 曲线框架, 讨论“关税率提高到一定程度后, 收入可能不再增加甚至下降”的经典结论, 继而结合美国历史收入数据构造静态与动态的收入情景。

相较于复杂的非线性机器学习模型, Laffer 型的传统函数形式在这里具有明显优势: 模型结构简单、经济含义直接, 便于向政策制定方解释“为何过高关税未必带来更多财政收入”。

\subsection{收入路径与 Laffer 曲线图示}

图~\ref{fig:q4_revenue} 展示了不同政策情景下未来若干年的关税收入路径; 图~\ref{fig:q4_laffer} 对不同弹性假设下的 Laffer 曲线进行了可视化。

\begin{figure}[H]
  \centering
  \includegraphics[width=0.8\textwidth]{../figures/q4_revenue_time_path.pdf}
  \caption{不同政策情景下关税收入时间路径}
  \label{fig:q4_revenue}
\end{figure}

\begin{figure}[H]
  \centering
  \includegraphics[width=0.7\textwidth]{../figures/q4_laffer_curve.pdf}
  \caption{关税率--收入 Laffer 曲线示意图}
  \label{fig:q4_laffer}
\end{figure}

通过这些传统框架, 我们在 Q4 中较好地平衡了“理论清晰度”和“情景可操作性”, 使得结论既便于理解又便于在政策讨论中引用。

\section{Q5 宏观金融: VAR + 深度混合的综合框架}

\subsection{VAR 与回归的宏观解释}

在 Q5 中, 我们首先构建了多条时间序列回归和 VAR 模型, 将关税指数、反制指数与 GDP 增速、工业生产、制造业增加值占比等关键宏观指标联系起来。传统 VAR/回归在此处的优势在于:
\begin{itemize}
  \item 提供了标准化的脉冲响应(IRF)视角, 便于分析“关税冲击”在若干期内对各变量的动态影响;
  \item 参数结构清晰, 可直接用于撰写文字化的宏观分析段落。
\end{itemize}

图~\ref{fig:q5_ts} 展示了关税指数、GDP 增速、制造业份额和美元指数的基本时间趋势, 为后续的模型分析提供直观背景。

\begin{figure}[H]
  \centering
  \includegraphics[width=0.85\textwidth]{../figures/q5_time_series_overview.pdf}
  \caption{关税指数与主要宏观、金融变量时间序列概览}
  \label{fig:q5_ts}
\end{figure}

在 VAR 模型的基础上, 我们进一步构造了以关税冲击为核心的脉冲响应分析, 如图~\ref{fig:q5_irf} 所示, 用以展示关税变化对宏观变量在若干期内的方向性冲击路径。

\begin{figure}[H]
  \centering
  \includegraphics[width=0.8\textwidth]{../figures/q5_impulse_response.pdf}
  \caption{关税冲击下主要变量的脉冲响应示意}
  \label{fig:q5_irf}
\end{figure}

\subsection{VAR--LSTM 与树模型: 综合利用深度与 ML 优势}

在 Q5 中, 我们进一步引入 VAR--LSTM 混合模型与树模型(Random Forest、Gradient Boosting), 对 VAR 残差与制造业回流指标进行建模, 体现如下优势:
\begin{itemize}
  \item 利用 LSTM 对 VAR 残差中的非线性与结构性变化进行捕捉, 在保持宏观可解释性的同时补充深度特征;
  \item 利用树模型对制造业增加值占比与回流 FDI 的驱动因素进行排序, 为后续政策分析提供特征重要性视角;
  \item 通过传统模型与深度/ML 模型的互相印证, 构建出一个“解释+预测”兼具的综合框架。
\end{itemize}

\subsection{模型性能的定量评估}

从宏观回归与 VAR 的结果看, 关税指数和反制指数对 GDP 增速、工业生产和制造业份额的平均解释度约为 $R^2 \approx 0.259$ (共 4 条回归), 在有限样本和高度聚合指标的前提下属于中等水平, 更适合作为方向性的参考而非高精度点预测。VAR 模型的稳定性检验显示系统并非严格意义上的完全稳定, 其脉冲响应图主要用于判断关税冲击在若干期内对各变量的方向和相对大小。进一步地, VAR--LSTM 混合模型在仅有约 7 条训练样本的条件下仍然可以将残差均方根误差控制在约 $0.708$ 的量级, 说明在给定结构骨架之后, 深度模型能够对复杂余项进行一定程度的细致刻画。对于制造业回流指标, 单独依赖树模型进行数值预测并非我们的主要目标, 它们更多被用来提供特征重要性排序和敏感性分析视角, 辅助我们识别“哪些金融与政策变量更值得重点关注”, 也进一步体现了综合框架在解释与预测之间的平衡思路。

总体而言, Q5 题展示了我们在宏观金融问题上综合运用 VAR、深度学习与机器学习的能力, 在复杂系统问题上形成了结构化解释与灵活预测相结合的分析路径。

\section{综合讨论与结论}

从整体上看, 本文在 Q1--Q5 中形成了如下互补格局:
\begin{itemize}
  \item Q1--Q2: 依托深度学习模型(LSTM、Transformer)的强表达能力, 在传统价格弹性与结构回归的基础上, 对复杂时间序列和策略空间进行了更细致的刻画;
  \item Q3--Q4: 发挥传统计量和结构化指标体系的稳定与可解释优势, 在供应链安全与关税收入问题上给出了清晰的理论与情景框架;
  \item Q5: 将 VAR、VAR--LSTM 和树模型自然结合, 在宏观金融与制造业回流问题上体现出多层次综合分析能力。
\end{itemize}

这种“传统 + 深度 + ML”的组合思路, 一方面确保了结论在经济意义上的清晰与可阐释, 另一方面也展示了我们对前沿模型的理解与运用能力, 为类似数据条件下的政策分析提供了一种可供借鉴的技术路线。

\section*{致谢}

感谢比赛组织方提供的数据框架与背景材料; 感谢指导老师和队友在数据清洗、模型设计和图表美化方面的支持。

\end{document}
